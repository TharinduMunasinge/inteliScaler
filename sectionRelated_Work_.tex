\section{Related Work}
Significant research has been conducted in the domain of cloud auto-scaling, particularly at IaaS level. However, we limit our focus to PaaS level and IaaS solutions that can be adopted to PaaS. A pluggable auto-scaling system that adds hardware and QoS awareness while capturing the cost of a scaling decision to complement AppScale PaaS is presented in \cite{pluggable}. Authors seek to provide an auto-scaling solution, which learns the behavior of a web application and provides optimal scaling decisions on AWS using hot spares and spot instances. Dependable Compute Cloud (DC2) is an application agnostic, model driven, adaptive auto-scaling system proposed in \cite{modeldriven}. DC2 employs a Kalman Filtering technique in combination with a queueing theoretic model to proactively scale resources according to the varying workload. DC2 addresses the important segment of auto-scaling, namely the removal of user input to specify scaling decisions. However, it does not capture the cost incurred by the scaling process and therefore is not a complete auto-scaling solution that is cost effective. SLA is an important factor when providing cloud resources as services. A SLA defines the contract between a service provider and a service consumer on an agreed QoS level. SLA-driven Cloud Autoscaling (SCAling) an advanced implementation of cloud elasticity based on SLA \cite{sladriven}. It successfully handles the trade-off between profit and customer satisfaction level without requiring manual intervention. The main idea is to exploit the SLA requirements to propose dynamic resource provisioning.\\

Further we adapted knowledge developed outside cloud computing domain, specially in the development of proposed prediction model. Single order auto regression methods are used for prediction by Kupferman et al. \cite{Kupferman_2009}. Results shows that prediction accuracy depends on parameters such as input window size and horizon window size. Quadratic exponential smoothing applied against real workload traces such as World Cup 98 \cite{WorldCup_1998} show accurate prediction \cite{Mi_2010}. Further Chris et al. \cite{Bunch_2012} used an exponential smoothing algorithm to forecast the expected and enqueued number of requests for the next $t$ seconds for a PaaS auto-scaler. Auto-Regressive Moving Average (ARMA) method is one of the dominant time series analysis techniques for workload and resource usage prediction. Roy et al. \cite{Roy_2011} used a second order ARMA filter for workload prediction on the World Cup 98 traces and showed accurate results. Yang et al. \cite{Yang_2013} have used use a sliding window based Linear Regression Model (LRM) for workload prediction and showed a lower prediction deviation. Hidden Markov Model (HMM) is used to explore the temporal correlations in workload pattern changes by Khan et al. \cite{Khan_2012}. Some work focused on using history window values as the input for a neural network \cite{Islam_2012}. However, the accuracy of such method depends on the input window size.\\