\section{Related Work}
Significant research has been conducted in the domain of cloud auto-scaling, particularly at IaaS level. However, we limit our focus to PaaS level and IaaS solutions that can be adopted to PaaS.

A pluggable auto-scaling system that adds hardware and QoS awareness while capturing the cost of a scaling decision to complement AppScale PaaS is presented in \cite{Bunch_2012}. Authors seek to provide an auto-scaling solution, which learns the behavior of a web application and provides optimal scaling decisions on AWS using hot spares and spot instances. While the paper emphasizes the importance of QoS factors as well as cost model awareness it lacks reliable workload prediction mechanism that resource allocation mechanism can rely on.

Dependable Compute Cloud (DC2) is an application agnostic, model driven, adaptive auto-scaling system proposed in \cite{modeldriven}. DC2 employs a Kalman Filtering technique in combination with a queueing theoretic model to proactively scale resources according to the varying workload. DC2 addresses the important segment of auto-scaling, namely the removal of user input to specify scaling decisions. However,it does not capture the cost incurred by the scaling process and therefore is not a complete auto-scaling solution that is cost effective.

SLA is an important factor when providing cloud resources as services. A SLA defines the contract between a service provider and a service consumer on an agreed QoS level. SLA-driven Cloud Auto-scaling (SCAling) an advanced implementation of cloud elasticity based on SLA \cite{sladriven}. It successfully handles the trade-off between profit and customer satisfaction level without requiring manual intervention. The main idea is to exploit the SLA requirements to propose dynamic resource provisioning.

Yang et al. \cite{Yang_2013} have used use a sliding window based Linear Regression Model (LRM) for workload prediction and showed a low prediction deviation. They have also proposed an auto-scaling mechanism to scale virtual resources at different resource levels in service clouds which combines real-time scaling and pre-scaling under three scaling techniques, namely self-healing scaling, resource-level scaling and VM-level scaling.

Simultaneous optimization of resource cost, QoS and availability is a major challenge in the context of cloud auto-scaling. Roy et al. \cite{Roy_2011} have addressed this challenge with a resource allocation algorithm based on model predictive techniques, which allocates or deallocates machines to applications with the goal of optimizing their utility over a limited prediction horizon. As part of their research, they have used a second order ARMA filter for workload prediction on the World Cup 98 traces \cite{WorldCup_1998} and showed accurate results.

Even though extensive research has been conducted on various aspects of cloud auto-scaling, we have recognized that available PaaS cloud solutions stick to threshold driven,rule based reactive auto-scaling. This is mainly due to the lack of a solution that addresses both future workload prediction and resource allocation.