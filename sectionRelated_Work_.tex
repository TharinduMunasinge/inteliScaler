\section{Related Work}
Significant amount of research has been conducted related in domain of cloud autoscaling. 
\cite{pluggable} focus on developing a pluggable autoscaling system that adds HA awareness and QoS awareness while capturing the cost of a scaling decision to complement AppScale PaaS. They seek to provide a auto scaling solution which learns the behavior of a web application and provides optimal scaling decisions on AWS using hot spares and spot instances.\\

Dependable Compute Cloud (DC2) is an application agnostic, model-driven, adaptive autoscaling system proposed in \cite{modeldriven}. DC2 employs a Kalman Filtering technique in combination with a queueing theoretic model to proactively scale resources according to the varying workload. DC2 addresses the important segment of auto scaling, namely the removal of user input to specify scaling decisions. However, it does not capture the cost incurred by the scaling process and therefore is not a complete auto scaling solution that is cost effective\\

SLA is an important factor when providing cloud resources as services. A SLA defines the contract between a service provider and a service consumer on an agreed QoS level. \cite{sladriven} presents SLA-driven Cloud Autoscaling (SCAling) an advanced implementation of Cloud elasticity based on SLA. It successfully handles the trade-off between profit and customer satisfaction level without requiring manual intervention. The main idea is to exploit the SLA requirements to propose dynamic resource provisioning.\\

Further we adapted knowledge developed outside cloud computing domain specially in the development of prediction model proposed. Single order auto regression methods are used for prediction in Kupferman et al. \cite{Kupferman_2009}. Results shows that prediction accuracy depends on parameters such as input window size and horizon window size. \cite{Mi_2010} applies quadratic exponential smoothing against real workload traces such as World Cup 98 \cite{WorldCup_1998}, and showed accurate prediction. Further Chris et al. used an exponential smoothing algorithm to forecast the expected and enqueued number of requests  for the next $t$ seconds \cite{Bunch_2012} for an auto scaler in PaaS cloud.Auto-Regressive Moving Average (ARMA) method is one of the dominant time series analysis techniques for workload and resource usage prediction. Roy et al. \cite{Roy_2011} used a second order ARMA filter for workload prediction on the World Cup 98 traces and showed accurate results.\\

Lot of research has been conducted on machine learning based prediction as well. Yang et al. \cite{Yang_2013} have used use a sliding window based linear regression model (LRM) for workload prediction and showed a lower prediction deviation.Hidden Markov Model (HMM) is used to explore the temporal correlations in workload pattern changes by Khan et al. \cite{Khan_2012}. Some  researches are focused on using history window values as the input for a neural network \cite{Islam_2012}. However, the accuracy of such method depends on the input window size.\\