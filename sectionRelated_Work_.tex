\section{Related Work}
Auto scaling is a widely studied research area in all levels and aspects of cloud computing. While there are many definitions in diverse perspectives from academics and cloud vendors, we find Gartner’s definition of auto scaling more accurate as it captures the overall impact of the process, as follows:\\\\

Auto-scaling automates the expansion or contraction of system capacity that is available for applications and is a commonly desired feature in cloud IaaS and PaaS offerings. When feasible, technology buyers should use it to match provisioned capacity to application demand and save costs.\cite{website:gartner}\\

Auto-scaling is often interchanged with the meaning of scalability, elasticity and resource provisioning, which is incorrect as they are different concepts \cite{autoscalingissues}. Most of the definitions depict auto-scaling as a measure of provisioning resources on demand to handle the required workload, ignoring the cost factor and not addressing the necessity of auto-scaling, which is to give complete autonomy to the system, lessening the burden  of managing resources from the user. Approaches taken to solve this problem can be classified according to the following techniques: Rule-based, Machine Learning, Queueing Theory and Control Theory. A comprehensive overview of these techniques and related work can be found in \cite{reviewofautoscaling}.\\

\cite{pluggable} focus on developing a pluggable autoscaling system that adds HA awareness and QoS awareness while capturing the cost of a scaling decision to complement AppScale PaaS. They seek to provide a auto scaling solution which learns the behavior of a web application and provides optimal scaling decisions on AWS using hot spares and spot instances. Work is ongoing to adapt the algorithms proposed in these works as auto scalers within the AppScale PaaS.\\

Dependable Compute Cloud (DC2) is an application agnostic, model-driven, adaptive autoscaling system proposed in \cite{modeldriven}. DC2 employs a Kalman Filtering technique in combination with a queueing theoretic model to proactively scale resources according to the varying workload. DC2 addresses the important segment of auto scaling, namely the removal of user input to specify scaling decisions. However, it does not capture the cost incurred by the scaling process and therefore is not a complete auto scaling solution that is cost effective.\\

SLA is an important factor when providing cloud resources as services. A SLA defines the contract between a service provider and a service consumer on an agreed QoS level. \cite{sladriven} presents SCAling – SLA-driven Cloud Auto-scaling – an advanced implementation of Cloud elasticity based on SLA. It successfully handles the trade-off between profit and customer satisfaction level without requiring manual intervention. The main idea is to exploit the SLA requirements to propose dynamic resource provisioning.\\

Our study has revealed that most of the existing Cloud Service Providers (CSP) offer auto scaling solutions on a rule based engine and typically require the user to specify threshold values on resource usage. Further, such numerical values need to be modeled and tuned to maximize resource utilization and minimize operational cost including acquisition cost and SLA violation cost. In contrast, inteliScaler address all these aspects by providing a complete solution.\\