\begin{table}[h!]
\centering
\caption{QoS summary for evaluation for Workload I on AWS EC2}
\label{table:analysis_qos_workload_3}
\begin{tabular}{|l|c|c|c|c|}
\hline

Test Case & Average & Requests & Requests & Time Out  \\
& Response Time & Initiated & Completed & Errors\\ \hline

low threshold & 6.3839 s & 254535 & 203067 & 2511\\ \hline

high threshold & 4.4954 s & 280477 & 191891  & 11863\\ \hline

inteliScaler & 5.1288 s & 286180 & 261979  & 2278\\ \hline

\end{tabular}
\end{table}

\subsubsection{Workload II}
We tested both Stratos and inteliScaler with a fluctuating yet growing workload as described in Table \ref{table:workload_5}.\\

As shown in Fig. 8, inteliScaler has the lowest resource allocation of 10 VMs whereas for the same workload Apache Stratos allocated 15 VMs, saving the cost for 5 VMs. Also the QoS of inteliScaler is better than Stratos as shown in Table \ref{table:analysis_qos_workload_5}. inteliScaler has responded to 94.24\% of the requests generated whereas Stratos has responded to only for 57.41\% of the total requests generated. There are significant number of time outs and drop off errors in Stratos compared to inteliScaler, which is the reason for these figures.\\

\begin{table}[h!]
\centering
\caption{Workload II for AWS setup.}
\label{table:workload_5}
\begin{tabular}{|l|l|l|l|l|l|l|l|l|l|l|}
\hline
Segment & 1 & 2 & 3 & 4 & 5 & 6 & 7 & 8 & 9 & 10\\ \hline
Duration (s) & 90 & 90 & 90 & 90 & 90 & 90 & 90 & 90 & 90 & 90 \\ \hline
Users (\times100) & 2 & 1 & 4 & 2 & 6 & 4 & 8 & 6 & 10 & 8   \\ \hline
Transition (s) & 30 & 30 & 30 & 30 & 30 & 30 & 30 & 30 & 30 & 30  \\ \hline
\end{tabular}
\end{table}