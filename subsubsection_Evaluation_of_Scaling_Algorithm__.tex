\subsubsection{Evaluation of Scaling Algorithm With Different Instance Sizes}
We evaluated the behavior of proposed scaling algorithm with different instance types.  
\begin{table}
\label{different_instance}
\centering
\begin{tabular}{|l|l|l|l|l|}
\hline
Units
& Resource Cost
& Violation Cost
& Total Cost
& Violation Percentage \%
\\ \hline
X
&\textbf{3.25} 
&0.235
&\textbf{3.485}
&9.26
\\ \hline
2X
&3.617
&0.240
&3.857
&3.738
\\ \hline
4X
&5.09
&\textbf{0.192}
&5.282
&\textbf{0.932}
\\ \hline
\end{tabular}
It can be observed from \ref{different_instance} that smaller the instance, lower the resource cost and total cost. But percentage of performance violations tends to be higher. On the other hand larger instance incur higher resource cost but lower performance degradation. At top level in the graph performance violation is cost therefore no  violation incurred.