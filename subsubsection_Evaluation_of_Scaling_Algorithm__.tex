\subsubsection{Evaluation of Scaling Algorithm With Different Instance Sizes}
We evaluated the behavior of proposed scaling algorithm with different instance types.

It can be observed from Table \ref{dif_instance} that smaller the VM instance, lower the resource cost and total cost. But percentage of performance violations tends to be higher. Alterntively, larger VM instances incur higher resource cost but lower performance degradation. Considering the added complexity in handling heterogeneous platforms, we entirely focused on homogeneous clusters when implementing the solution on Apache Stratos.

\begin{table}
\caption{Behaviour of proactive solution with different VM instance types.}
\centering
\label{dif_instance}
\begin{tabular}{|c|c|c|c|c|}
\hline
Units
& Resource Cost
& Violation Cost
& Total Cost
& Violation Percentage
\\ \hline
X
&\textbf{3.25} 
&0.235
&\textbf{3.485}
&9.26
\\ \hline
2X
&3.617
&0.240
&3.857
&3.738
\\ \hline
4X
&5.09
&\textbf{0.192}
&5.282
&\textbf{0.932}
\\ \hline
\end{tabular}
\end{table}