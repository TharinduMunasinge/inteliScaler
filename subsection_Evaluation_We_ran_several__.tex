\subsection{Evaluation}

We ran several workloads, with drastically different characteristics, on the actual implementation of Apache Stratos and inteliScaler deployed in AWS EC2 setup.

\subsubsection{Workload I}
First we tested the same workload that was used earlier as shown in Table \ref{table:analysis_workload} against inteliScaler. Configurations of Stratos are given in Table \ref{table:policy_threshold}.\\

It is clear that inteliScaler outperforms both the configurations of Stratos as shown in Fig. 7 inteliScaler has allocated less number of VM's to handle the same workload. In terms of QoS, inteliScaler has a success ratio of 91.54\%, which is significantly higher than the 80\% of low threshold configuration and 68\% of high threshold configuration. Number of time out errors is also significantly low where in inteliScaler it is 0.8\%, while low threshold configuration has 0.9\% and high threshold configuration has 4.2\%.\\

\begin{table}[h!]
\centering
\caption{Workload used for AWS performance analysis.}
\label{table:analysis_workload}
\begin{tabular}{|l|l|l|l|l|l|l|}
\hline
Segment & 1 & 2 & 3 & 4 & 5 & 6\\ \hline
Duration (seconds) & 240 & 240 & 240 & 240 & 240 & 240 \\ \hline
Users & 400 & 800 & 1600 & 3200 & 800 & 400 \\ \hline
Transition (seconds) & 30 & 30 & 30 & 30 & 30 & 30 \\ \hline
\end{tabular}
\end{table}

\begin{table}[h!]
\centering
\caption{Auto-scaling policy of Stratos for different configurations.}
\label{table:policy_threshold}
\begin{tabular}{|c|c|c|c|}
\hline
Setup & Load Average & Memory Consumption & Request in Flight \\ \hline
Low Threshold & 70 & 70 & 100\\ \hline
High Threshold & 95 & 95 & 150\\ \hline
\end{tabular}
\end{table}
