\section{Experimental Results}
Here we evaluate inteliScaler within Apache Stratos deployed on top of AWS EC2. We begin by presenting our experimental methodology, and discuss the results. \\

\subsection{Experimental Setup}
In order to evaluate the scaling process of Apache Stratos we cannot simply use either a simulation or the built-in mock IaaS feature, since they do not capture every aspect of a real system such as fault detection and handling, spawn time delay and in-flight requests count. Hence we deployed Apache Stratos 4.1.4 on AWS EC2 in a distributed setup to run a few workloads using the RUBiS benchmark and allow the system to scale using default policies. Configuration details of the setup are given in table \ref{tab:stratos_config}. 


\begin{table}[h!]
\centering
\caption{Apache Stratos setup configuration}
\label{tab:stratos_config}
    \begin{tabular}{ | p{1cm} | p{2cm} | p{1cm}| p{4cm}|}
    \hline
    Stratos Component & Implementation & EC2 Instance Type & Details \\ \hline
    Stratos & Apache Stratos 4.1.4 & t2.medium & This includes Stratos Manager and the Auto-Scaler \\ \hline
    
    Message Broker & Apache ActiveMQ 1.8 & t2.medium & Used to communicate between each component and Cartridge Agent on each node \\ \hline
    
    Load Balancer & HAProxy 1.6 & t2.medium & One static load balancer to handle application requests \\    \hline
    
    Complex Event Processor & WSO2 CEP 3.1 & r3.large & Aggregates all health stats gathered from each node real-time and produces average, gradient, derivative values \\    \hline
    
    Database & MySQL & r3.large & Central database for all nodes to read/write \\    \hline
    
    Load Generator & Rain Tool Kit & r3.large & Creates multiple connections with the Load Balancer to emulate the stipulated users and their actions \\    \hline
    \end{tabular}
    
\end{table}

At the time of evaluation, the latest stable version of Apache Stratos was 4.1.4. Apache ActiveMQ was chosen as it is the message broker recommended by Apache Stratos, although other brokers like RabbitMQ are also known to be supported. We tried almost all the options for the load balancer that Apache Stratos supported (according to their documentation), but we could only manage to set up HAProxy as the load balancer with a few modifications on our own. WSO2 CEP was the only choice for the event processor as the implementation is heavily coupled with the architecture of WSO2 CEP. We used a central database and scaled only the PHP instances, avoiding the trouble of syncing databases across multiple nodes which would have been the case if a MySQL was also provisioned on a cluster. For the central database we had two setups, one using AWS RDS and the other with our own dedicated server. Since AWS RDS had certain limitations such as the maximum number of concurrent connections, we set up our own node by tuning its performance to handle up to 5000 concurrent connections. Two extra nodes were set up on the same network to enumerate the workload for the RUBiS app deployed on top of Apache Stratos.