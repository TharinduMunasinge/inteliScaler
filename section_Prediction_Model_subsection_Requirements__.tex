\section{Prediction Model}\\

\subsection{Requirements}
While proactive auto scaling has can be beneficial for a PaaS with diverse applications and drastically differing resource demands, such auto scalers should be capable of accurately predicting future workloads over a reasonable time horizon in order for their proactive decisions to be effective.
Making early scaling decisions, such that the required resources would be allocated in advance to cater for future workloads, is highly dependent on the availability of such accurate predictions.

The following can be identified as some major characteristics of a good cloud workload prediction model:

\begin{itemize}
\item Ability to predict accurately over a sufficiently large time horizon, considering factors like VM start-up and shutdown delays
\item Online training capability, as evolution and continuous learning of new workload characteristics is essential as the workload dataset grows with time
\item Bounded processing time per auto scaling request
\item Immunity to overfitting to specific patterns, since it should be able to deal with different applications and hence a wide variety of dynamic workload patterns
\end{itemize}

\subsection{Approach}
The time-variant metrics practically employed in measuring cloud workloads, such as CPU utilization, memory consumption, and in-flight request count can effectively be treated as time series, transforming the prediction problem to the more generalized notion of time series analysis.

